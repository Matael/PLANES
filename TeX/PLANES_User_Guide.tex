\documentclass[11pt,A4paper]{book}
\usepackage{geometry}        
 
\usepackage[parfill]{parskip}  
\usepackage{amssymb}
\usepackage{epstopdf}



\usepackage[pdftex]{graphicx}         % graphiques de base
% Include figure files
%\usepackage{dcolumn}% Align table columns on decimal point
\usepackage{amsmath,amsfonts}%
\usepackage{bm}%
\usepackage{psfrag}%
\usepackage{pict2e}


\DeclareGraphicsRule{.tif}{png}{.png}{`convert #1 `dirname #1`/`basename #1 .tif`.png}

\usepackage[colorlinks=true, pdfstartview=FitV, linkcolor=blue, 
            citecolor=blue, urlcolor=blue]{hyperref}

\usepackage{dirtree}
%\includeonly{Chapter1}

% Nouvelles commandes
\newcommand{\ds}{\displaystyle}
\newcommand{\tb}{\textbf}
\newcommand{\tr}{\textrm}
\newcommand{\bs}{\boldsymbol}
\newcommand{\tp}{^T}
\newcommand{\mb}{\, \mathbb}
\newcommand{\pt}{\textbf{.}}
\newcommand{\esp}{, \quad}
\newcommand{\D}{\textrm{D}}
\newcommand{\p}{\partial}

\newcommand{\virt}[1]{\widehat{\textbf{#1}}}
\newcommand{\To}{\longrightarrow}
\newcommand{\abs}[1]{ \left\vert #1 \right\vert}
\newcommand{\Gd}[1]{\,\boldsymbol{\nabla}#1\,}
\newcommand{\Div}{\,\boldsymbol{\nabla}\textbf{.}}
\newcommand{\gd}[1]{\,{\nabla}#1\,}
\renewcommand{\div}{\, \nabla \textbf{.}}
\newcommand{\Rot}{\,\boldsymbol{\nabla}\wedge}
\newcommand{\rot}{\,\nabla \wedge}
\newcommand{\wdt}[1]{\widetilde{#1}}
\newcommand{\wtt}[1]{\breve{#1}}
\newcommand{\cv}[1]{\ast}
\renewcommand{\d}{\,\textrm{d}}
\newcommand{\phys}[1]{\widetilde{#1}}
\newcommand{\Four}[1]{\mc{F}\left(#1\right)}
\newcommand{\Fouri}[1]{\mc{F}^*\left(#1\right)}


\newcommand{\oiom}{\oint_{\partial \Omega}}
\newcommand{\iom}{\int_{\Omega}}
\newcommand{\iome}{\displaystyle{\int_{\Omega_e}}}

\newcommand{\oiomO}{\oint_{\partial \Omega_0}}
\newcommand{\iomO}{\int_{\Omega_0}}
%\newcommand{\scal}[1][2]{\left<#1|#2\right>}
\newcommand{\dev}[1]{\widehat{#1}}

\newcommand{\mat}[1]{#1}
\renewcommand{\mat}[1]{ \left[ \textbf{#1} \right]}
\renewcommand{\:}{\tb{:}}
\newcommand{\chapterm}[1]{\markboth{\chaptername \thechapter.
 #1}{}}

\newcommand{\np}{\newpage{\cleardoublepage}}
\newcommand{\be}{\begin{equation}}
\newcommand{\ee}{\end{equation}}
\newcommand{\lb}[1]{\label{#1}}
\newcommand{\refeq}[1]{(\ref{#1})}

\newcommand{\w}[1]{textbf{#1}}

\newcommand{\cj}{\overline}
 \newcommand{\norm}[1]{\left\Vert#1\right\Vert}
 \newcommand{\essnorm}[1]{\norm{#1}_{\ess}}
\newcommand{\eps}{\varepsilon}
\newcommand{\re}[1]{\Re{(#1)}}
\newcommand{\im}[1]{\Im{(#1)}}
\newcommand{\scal}[2]{\left<#1|#2\right>}
% Fin nouvelles commandes

\newcommand{\espa}{\vspace*{10cm}}
\renewcommand{\mat}[1]{#1}
\newcommand{\mc}{\mathcal}
\newcommand{\ov}{\overline}
\newcommand{\etal}{\textit{et al.}\,}
%Headers
\newcommand{\Hunc}{\mc{H}^1(\Omega)}
\newcommand{\Huncp}{\mc{H}^1_P(\Omega)}
\newcommand{\st}{^*}
\newcommand{\goodgap}{\hspace{\subfigtopskip}}

\newcommand{\ut}{$\{\tb{u}^s,\tb{u}^t\}\,$}
\newcommand{\uU}{$\{\tb{u}^s,\tb{u}^f\}\,$}
\newcommand{\uP}{$\{\tb{u}^s,P\}\,$}
\newcommand\flash[1]{\colorbox{yellow}{\textbf{#1}}}


\setlength{\unitlength}{1cm}

% ------------------- Title and Author -----------------------------
\title{\tb{PLANES Users Guide}}
\author{Olivier DAZEL\\
LAUM UMR CNRS 6613\\
olivier.dazel@univ-lemans.fr}
\begin{document}


\maketitle

\tableofcontents


\chapter{General overview}

\section{Presentation}

PLANES (Porous LAum NumErical Simulator) project is a collection of Matlab/Fortran scripts to simulate the vibroacoustics response of coupled systems including acoustic, elastic, porous materials, PML... 

\section{General organisation}

\dirtree{%
.1 PLANES.
.2 Projects.
.3 m.
.4 Materials.
.3 out.
.3 FF.
.2 src.
.3 Main.
.2 TeX.
}
All the information relative to a project are in the \verb?Projects? folder. As a user, you do not need (theoretically) to modify the files in the other folders. 

\subsection{Labels for materials \label{sec:label_materials}}

The convention is as follows:  
  
  
\begin{itemize}
\item 0 AIR
\item 1XXX Elastic medium
\item 2XXX Equivalent fluid (rigid frame) material
\item 3XXX Limp model
\item 4XXX Poroelastic Material (or FEM: 1998 formulation)
\item 5XXX BIOT (or FEM: 2001 formulation) 
\item 80xy PML x and y boolean direction
\end{itemize}


\subsection{Type of boundaries}
\begin{itemize}
\item 1 RIGID WALL
\item 2 UNIT PRESSURE (FLUID)
\item 3 UNIT NORMAL VELOCITY
\item 4 UNIT TANGENTIAL VELOCITY
\item 5 SLIDING (PEM)
\item 6 BONDED (PEM) or CLAMPED (elastic)
\item 7 UNIT PRESSURE (PEM)
\item 8 UNIT NORMAL VELOCITY (PEM)
\item 9 UNIT NORMAL VELOCITY (PEM)
\item 10 INCIDENT AIR PLANE WAVE on  ACOUSTIC/Biot98  ELEMENT 
\item 11 INCIDENT AIR PLANE WAVE on ELASTIC ELEMENT 
\item 12 INCIDENT AIR PLANE WAVE on Biot2001 ELEMENT
\item 13 DtN Plate
\item 20 TRANSMITTED AIR PLANE WAVE on  ACOUSTIC/Biot98  ELEMENT 
\item 21 TRANSMITTED AIR PLANE WAVE on ELASTIC ELEMENT 
\item 21 TRANSMITTED AIR PLANE WAVE on Biot2001 ELEMENT
\item 60 UNIT NORMAL VELOCITY on H12 with FLUX APPLICATION
\item 61 DGM RADIATIVE BOUNDARY
\item 98 PERIODICITY LEFT
\item 99 PERIODICITY RIGHT
\item 4xx ZOD impair/pair
\item 400 FSI
\item 1xyz Excitation wave 1 angle xyz in degree(PEM)
\item 2xyz Excitation wave 2 angle xyz in degree(PEM)
\item 3xyz Excitation wave 3 angle xyz in degree(PEM)
\item 500 VELOCITY DIFFRACION CYLINDRE EF
\end{itemize}


\subsection{Models for elements}
\begin{itemize}
\item 1 TR6
\item 2 H12
\item 3 TR3
\item 10 DGM on TR
\item 11 DGM on H
\end{itemize}



\chapter{How to use the Multilayer solver}


The Multilayer solver can be launched with a call

\begin{verbatim}PLANES_Multilayer(Name,Number,data_model,multilayer_1,frequency)
\end{verbatim}

\begin{itemize}
	\item \verb?Name? is a string associated to the name of the project. This string is the same than the folder in The project area.
	\item \verb?Number? is an integer. This is the number of the subproject
	\item \verb?data_model? is a structure that contains the data of the model. In the present case, it only contains the angle of incidence.
	\item \verb?multilayer? is a structure array which contains one or several multilayer structures. It is associated to several arrays
	\begin{itemize}
	\item \verb?multilayer.nb(1,#m)? correspond to the number of layer of multilayer structure\verb?#m?.
	\item \verb?multilayer.termination(1,#m)? correspond to the termination condition of multilayer structure\verb?#m?. The value is 0 for rigid backing and 1 for a radiation condition
	\item \verb?multilayer.d(#l,#m)? correspond to the thickness of the layer \verb?#l? of multilayer structure \verb?#m?. It is a real number.
	\item \verb?multilayer.mat(#l,#m)? correspond to the material of the layer \verb?#l? of multilayer structure \verb?#m?. The label is an integer associated to the convention presented in section \ref{sec:label_materials}.
	\end{itemize}
		\item \verb?data_model? is a structure that contains the data of the model. In the present case, it only contains the angle of incidence. It can be either a real number or an array of two real numbers. 
	\item  \verb?frequency? is the structure associated to frequency 
\end{itemize}


The result of the Multilayer solver is a File:
\begin{center}
	\begin{verbatim}
		Name of the project_#subproject.PW
	\end{verbatim}
\end{center}
It is a text file in which each line has $1+6l$ columns. The first one correspond to the frequency. The remaining $6l$ columns correspond to the result for multilayer structure $\#l$ and are ordered
\begin{itemize}
	\item Absorption coefficient
	\item Real part of the reflexion coefficient
	\item Imaginary part of the reflexion coefficient
	\item Transmission loss
	\item Real part of the transmission coefficient
	\item Imaginary part of the transmission coefficient
\end{itemize}


\chapter{How to use the FEM/DGM solver}


The result of the Multilayer solver is a File:
\begin{center}
	\begin{verbatim}
		Name of the project_#subproject.PL
	\end{verbatim}
\end{center}





\chapter{Benchmarks}

\section{Kundt Tube}

Sous projets
\begin{itemize}
	\item 0: TR6
	\item 1: H12
	\item 2: TR6/H12
	\item 3: DGM on TR
	\item 4: DGM on H
	\item 5: DGM on TR / DGM on H
	\item 6: H12 / DGM on H
	\item 7: TR6 / DGM on H  
\end{itemize}

\begin{figure}[BTBP]
\centering
\setlength{\unitlength}{1cm}
\begin{picture}(6,6)
\linethickness{1mm}
   \put(0,1){\line(1,0){4}}
   \put(0,2){\line(0,1){4}}
   \put(0 ,6){\line(1,0){4}}
   \put(0 ,2){\line(0,1){4}}
   \put(0.3,4){0}
\end{picture}
\caption{caption}
\end{figure}

\bibliographystyle{plain-annote}
\bibliography{mybibliography}


\end{document}

